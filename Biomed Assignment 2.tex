\documentclass[12pt]{article}
\author{Bhumika Dewangan}
\date{}
\title{EVOLUTION OF MODERN HEALTH CARE SYSTEM }
\begin{document}
\maketitle
\section{INTRODUCTION}
Healthcare in ancient times has been documented in vedas.Medical knowledge has been imparted to us by our ancestors. Modern medicine was first brought in by the Portuguese but the first hospitals were built by the British and French. 19th century saw organised training in allopathy and other medical streams.Increasing mobile and internet penetration has seen the growth of online pharmacies and online doctor consultation. Doctor and hospital reviews and online appointment booking systems have made it easier for patients to get desirable treatment. Behavioural and mental healthcare have also leveraged technology to disseminate information on topics that were considered a taboo for a very long time. With online information, first level screenings and counselling, primary care is easily provided to those who are somewhat aware of these increasing lifestyle issues.
\section{TECHNOLOGICAL ADVANCEMENTS}
\begin{itemize} 
\item Telemedicine has revolutionised the healthcare industry by ending distance and time constraints.Major benefits of Telemedicine include improved access to healthcare, cost effectiveness, improved quality and patient demand, especially in rural and remote areas while its constraints are lack of infrastructure and reimbursement from the third-party payers.
\item 
Touchscreens enhances usability in hospital settings ,offers innovative user interface,helps in easy navigation and 3D diagnostic models.
\item 
Voice recognition technology has reduced the graph of ignored reports.
\item 
With electronic health record (EHR)record can be accessed at one click and can become part of digital medical record,personal health record improve patients tracking ,encourage patients participation and offers social networking integration by allowing patients to talk to each others.Implementation of EHR would have data repositories that would house a vast amount of literature on diagnostics and treatments of diseases and disorders to help Physician to confirm diagnosis and decide the best treatment and method of healthcare delivery. This would shift the emphasis from opinion-based medical practice to evidence-based medical practice.
\item 
A clinical decision support (CDS) point a provider to the reference material and information,identify possible risks for adverse events and errors,raise alerts and provide reminders,encourage adherence to standards,analyze clinical performance.
\item
Cloud computing can provide efficiency and performance as its servers are virtualized, different instances can reside on the same hardware and also moved around depending on the need to make the best use of hardware without compromising performance.
\item 
Augmented reality has changed the way we see the the disease and organ modelling. Paediatric surgery possible accurately as augmented reality does precise mapping of pulmonary vessels,accurate surgical plans could be developed in less time.Through augmented reality we can not only just access pictures but can also perform things virtually which seems like reality.
\end{itemize}
\section{CONCLUSION}
we have reached a safe side to save humanity but we have lot more to go.WE lack proper disease modellingdue to which it is not possible to decide the targeted treatment that leads to duplication of efforts and hence increases the cost of medical system.
 
\end{document}