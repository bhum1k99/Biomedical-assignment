\documentclass[12pt]{article}
\usepackage{graphicx}
\graphicspath{{images/}}
\date{}

\begin{document}

\tableofcontents
\clearpage

\section{PULSE OXIMETER}

\begin{figure}
\centering
\includegraphics[scale=0.50]{pulse.jpg}
\end{figure}
\subsection{Introduction}

Pulse oximetry is a non invasive method for monitoring a person's oxygen saturation.It can measure the level of oxygen within the blood, specially in arterial blood without using any invasive means.It is an easy,painless way to find how well oxygen is being sent to parts of body.The human body requires and regulates a very precise and specific balance of oxygen in blood.Normal arterial blood oxygen saturation levels in human are 95 to 100 percent.If the level is below 90 percent, it is considered low and called hypoxemia.\\
\subsection{Way of using device}
While using device manufacturer's instruction should be read well,user should be in resting position preferably sitting,hand should be warm and relaxed,index finger without any deformity should be under red light and motion should be avoided to get precise results,we should wait untill a stable reading.
\subsection{Clinical usage}
Pulse oximeters may be used in a variety of situations that call for monitoring oxygenation and
pulse rates. Pulse oximeters increase patient safety by alerting the hospital staff to the onset of
hypoxia during or following surgery. Oximeters confirm adequate oxygenation during
mechanical ventilation. Physician and dental offices utilize pulse oximetry for spot checking
respiratory status, as well as for monitoring during procedures that call for sedation.Its a portable device making oxygen saturation testing available to the tips of patients hence shrinking distance and time barrier.  Truly, pulse
oximetry is the fifth vital sign, essential to complete patient monitoring.

\subsection{Working}

Pulse oximeter sensors have red and infrared low voltage
light emitting diodes (LEDs) wich serve as light sources. The
emitted light is transmitted through the tissue, then detected
by the photodetector and sent to the microprocessor of the
pulse oximeter. All constituents of the human
body, venous and arterial blood, and tissue absorb light. The pulsating of arterial blood results in changes
in the absorption to to added hemoglobin (Hb) and
oxygenated hemoglobin (HbO2) in the path of the light.
Since HbO2 and Hb absorb light to varying degrees, this
varying absorption is translated into plethysmographic
waveforms at both red and infrared wavelengths.
The relationship of red and infrared plethysmographic signal
amplitude can be directly related to arterial oxygen
saturation. 
\subsection{limitations}
There is a risk for inaccurate measurements that may result in undetected low oxygen saturation levels.Inspite of considering the the absolute threshold,measures over time should be taken unser consideration. Pulse oximeter accuracy is highest at saturations of 90-100 percent, intermediate at 80-90 percent, and lowest below 80 percent.There are accuracy difference between light and dark skin pigmentations,this differences are not negligible when saturation falls below 80 percent.
\section{INFUSION PUMPS}
\begin{figure}
\centering
\includegraphics[scale=0.80]{infusion pump.jpg}
\end{figure}
An external infusion pump is a medical device used to deliver fluids into a patient’s body in a controlled manner. There are many different types of infusion pumps, which are used for a variety of purposes and in a variety of environments.Infusion pumps may be capable of delivering fluids in large or small amounts, and may be used to deliver nutrients or medications – such as insulin or other hormones, antibiotics, chemotherapy drugs, and pain relievers.
\subsection{Working principle}
It uses pumping action to infuse fluids,medication or nutrients into patients body.It is suitable for intravenous,subcutaneous enteral and epidural infusions.
\subsection{Advantages}

Some infusion pumps are designed mainly for stationary use at a patient’s bedside. Others, called ambulatory infusion pumps, are designed to be portable or wearable.In general, an infusion pump is operated by a trained user, who programs the rate and duration of fluid delivery through a built-in software interface. Infusion pumps offer significant advantages over manual administration of fluids, including the ability to deliver fluids in very small volumes, and the ability to deliver fluids at precisely programmed rates or automated intervals.   They can deliver nutrients or medications, such as insulin or other hormones, antibiotics, chemotherapy drugs, and pain relievers. 
\subsection{limitations}
There are some risks associated with device:\\
\begin{itemize}
\item Alarming limitations:The infusion pump fails to generate an audible alarm for a critical problem, such as an occlusion (e.g., clamped tubing) or the presence of air in the infusion tubing.
The infusion pump generates an occlusion alarm in the absence of an occlusion.\\
\item Inadequate user interface design:The infusion pump screen doesn’t make clear which units of measurement the user is expected to enter.\\
\item delicate components:The infusion pump may have been dropped or damaged during use, which may result in an over-infusion or an under-infusion if the pump continues to be used without being repaired.\\
The plastic casing of an insulin pump, although promoted as waterproof, is prone to cracking, allowing water to enter the case and to cause the pump to malfunction. \\
\item Battery failure: A design issue causes over-heating of the battery and leads to premature battery failure. See photo on right.
\end{itemize}
\subsection{Types of infusion pumps}
\begin{itemize}
 \item Types according to the infusion pump’s mobility:\\
 \\
 Ambulatory Infusion Pump (AIP):These infusion pumps have a low weight that is often utilized to treat patients with debilitating problems.\\
 \\
ii) Stationary Infusion Pumps:Chronically ill patients who are bedridden for long periods of time usually need this type of medication infusions.\\
\item Types according to the pump capability of the fluid volume delivery:\\
\\
i) Syringe Pump:This pump has the capability to supply a low volume of the drug that you want.\\
\\
ii) Large Volume Pump (LVP):It is an infusion pump that can inject large amounts of medications and nutrients into the body of the patient.\\
\item Types according to the function of the pump:\\
\\
i) Specialty Pump:This pump mainly uses for home care services or treating a specific disease (for example, diabetes).\\
\\
ii) Conventional Infusion Pump:This pumps can be used for medical setting such as mobile, home, and long-term settings.\\
\end{itemize}
\clearpage

\section{UTERINE ASPIRATOR}
\begin{figure}
\centering
\includegraphics[scale=0.25]{aspirator.jpg}
\end{figure}
Vacuum or suction aspiration is a procedure that uses a vacuum source to remove an embryo or fetus through the cervix. The procedure is performed to induce abortion, as a treatment for incomplete miscarriage or retained pregnancy tissue, or to obtain a sample of uterine lining (endometrial biopsy). Uterine aspirators are intended to provide constant high-vacuum suction (>400 mmHg) and high flow through a fine curette to evacuate uterine contents from the uterus through the cervix. Uterine aspiration is used primarily as a method of induced abortion for terminating early pregnancies, treatment of incomplete spontaneous abortions or miscarriage, and removal of retained matter afterbirth. Vacuum for uterine aspiration can be produced either manually with a handheld syringe or an electric pump. An electrically operated aspirator consists of a vacuum-powered suction pump, a cannula, tubing,specimen container, a vacuum gauge, a vacuum control knob, an overflow trap, a moisture filter, and preferably a microbial filter. The uterine aspirators find applications in the field of gynaecology, obstetric, and surgery.Complete procedure generally takes 15 minutes to complete.
 \subsection{Clinical usage}
 Vacuum aspiration may be used as a method of induced abortion, as a therapeutic procedure after miscarriage, to aid in menstrual regulation, and to obtain a sample for endometrial biopsy. It is also used to terminate molar pregnancy.

When used as a miscarriage treatment or an abortion method, vacuum aspiration may be used alone or with cervical dilation anytime in the first trimester (up to 12 weeks gestational age). For more advanced pregnancies, vacuum aspiration may be used as one step in a dilation and evacuation procedure. Vacuum aspiration is the procedure used for almost all first-trimester abortions in many countries.

 \subsection{Advantages} Manual vacuum aspiration is the only surgical abortion procedure available earlier than the sixth week of pregnancy. Vacuum aspiration has lower rates of complications when compared to Dilation and Curettage.

Vacuum aspiration, especially manual vacuum aspiration, is significantly cheaper than sharp Dilation and Curettage. The equipment needed for vacuum aspiration costs less than a curette set. While sharp Dilation and Curettage is generally provided only by physicians, vacuum aspiration may be performed by advanced practice clinicians such as physician assistants and midwives.

Manual vacuum aspiration does not require electricity and so can be provided in locations that have unreliable electrical service or none at all. Manual vacuum aspiration also has the advantage of being quiet, without the louder noise of an electric vacuum pump.
\subsection{limitations}
It is not 100 percent effective in removing all uterine contents.The leftover content usually requires second aspiration procedure.There are chances of infections also perforation may cause injury to other internal organs.
\section{ECG MACHINE}
\begin{figure}
\centering
\includegraphics[scale=0.30]{ecg.jpg}
\end{figure}
An electrocardiogram (ECG)is a simple test that can check your heart rhythm and electrical activity.Sensors attached to the skin are used to detect electrical signals produced by your heart each time it beats.
\subsection{Parts of ECG}
1)Electrocardiography\\2)Electrocardiogram\\3)ecg paper\\4)ecg leads
\subsection{Working}
The elctrodes are connected to ECG machines by lead wires.the electrical activities of heart is then measured,interpreted and printed out.No electricity is sent to body.Natural electrical impulses coordinates contractions of different parts of heart to keep blood flowing the way it should.
\subsection{Uses}
\begin{itemize}
\item Helps diagnose and monitor conditions affecting heart.\\\item It can be used to investigate symptoms of possible heart problem,such as chest pain, palpitations,dizziness and shortness of breath.\\
\item 
An ECG can help detect:\\
i)arrhythmias – where the heart beats too slowly, too quickly, or irregularly\\
\\
ii)coronary heart disease – where the heart's blood supply is blocked or interrupted by a build-up of fatty substances
heart attacks – where the supply of blood to the heart is suddenly blocked\\
\\
iii)cardiomyopathy – where the heart walls become thickened or enlarged
A series of ECGs can also be taken over time to monitor a person already diagnosed with a heart condition or taking medication known to potentially affect the heart.
\end{itemize}
\subsection{advantages}
1)ECG is helpfil to measure three basic parameters of clinical interest that is rhythm and heart rate, axis of the heart and state of myocardial muscle.\\2)ECG represents data in topographic form which provides higher diagonostical information.\\3)ECG helps to prevent heart attacks by analysing heart parametes at initial stage.\\4)ECG can detect the cardiac conditions of the patients after surgical or any other operation and after application of anesthesia.
\subsection{Limitations}
1)It doesnot provide underlying heart problems for patients not having any symptoms.\\2)It doesnot always provides help in accurate diagnosis.More tests are needed to trace serious heart problems undetected by normal ECG curves.
\section{Spectrophotometer}
\begin{figure}
\centering
\includegraphics[scale=0.25]{spec.jpg}
\end{figure}
\subsection{introduction}
A spectrophotometer is a device that precisely measures electromagnetic energy at specific wavelengths of lights. It uses the characteristics of light and
energy to identify colors and determine how much of each color is present in
a ray of light.
\subsection{working}
The basic way a spectrophotometer functions is based on the absorption of
photons. Higher amounts of photons correspond to higher intensities of light.
Light is a form of electromagnetic radiation, like microwaves and gamma rays. The colors of the rainbow follow the progression of energy, with red
being the lowest and violet being the highest.\\
• “Spectro” refers to the fact that light is dispersed into individual or
groups of wavelengths in the electromagnetic spectrum of energy. Some
of that energy is in the ultraviolet and visible light range, which certain
spectrophotometers can read, while others can measure infrared radiation. A typical spectrophotometer can measure 31 wavelength bands
of light across a 300nm-wide range. More expensive instruments can
measure more than 150 bands of light across an 800nm-wide range.\\
• “Photometer” in the name refers to the ability to measure the intensity
of light at each group of wavelengths and scale it to a range of human
perception from 0-100. Zero equals total darkness and 100 is perfect
white. Some properties, like fluorescence, make it possible for this scale
to go over 100, so most spectrophotometers can reach 150 or 200.
By combining these two tools, we can generate specific data about the released colours and associated wavelengths to inform various applications.
\subsection{uses}

• Beverages: Color can indicate quality in many beverages from soft
drinks and juices to spirits and beer, and consistent color is critical to
inspire confidence in customers.\\
• Pharmaceuticals: The color of a pill is an integral part of identification.
It may not affect its functioning, but it tells people what they’re using. Other pharmaceutical products, like liquid ingredients, have strict
standards to meet, some of which involve its color and transparency.
Spectrophotometry helps ensure brand colors and identify counterfeit
medications.\\
• Chemicals: Chemicals must be clean, consistent in color and free of
contaminants to ensure proper functionality and that your customer
trusts them. Color is a key part of classifying many chemical products
and identifying their composition.\\
• Food: Food production uses spectrophotometry in many ways. From
evaluating the ripeness of fruits to identifying the appropriate baking
contrast of breads and buns, color analysis lends itself to plenty of
food-based applications.\\
• Molecular Biology: Applications for spectrophotometry include measurement of substance concentration such as protein, DNA or RNA,
growth of bacterial cells, and enzymatic reactions.
These are just a few examples, but you can find spectrophotometers in a lot
of industries having many applications, including uses outside of production,
like in vital biological research.
\subsection{limitation}
 You must clear the area of any outside light, electronic noise, or other outside contaminants that could interfere with the spectrometer's reading.


\end{document}
